% A Japanese Translation of "Purity of Election" (C.L.Dodgson)
% by Keisuke HARA
% Encoding: utf-8

\documentclass{article}

\title{純正なる選挙
    \footnote{この翻訳文書およびそのソースコードは
    Creative Commons CC BY 2.1 JP に従う。
    すなわち、著作者(原啓介)のクレジットを表示する限りにおいて、
    自由に本作品を複製、頒布、展示、実演、二次的著作物の作成が可能で、
    また、本作品を営利目的で利用することができる。}
    \footnote{pdf ファイル https://dl.dropboxusercontent.com/u/3255303/purity.pdf}
    \footnote{ソースコード https://github.com/kshara/Dodgson\_translations/}
    \\
    \medskip
    {\large
        ``Purity of Election"\footnote{
            {\it St.James's Gazette}, pp.4--6 (4 May 1881).
            See also {\it The Lewis Carroll Handbook},
            S.H.Williams, F.Maden, and R.L.Green.
            revised ed. by D.Crutch (1979).
            }
     \\
    (by Lewis Carroll)
    }
}
\author{訳: 原 啓介}

\begin{document}

\maketitle

\abstract{訳者より:
    ルイス・キャロルが {\it St.James's Gazette} 誌に
    投稿した、選挙改革についての提案。
    キャロル(C.L.ドジソン)が公正に投票者の意見を反映できるような選挙方法の
    数学的アルゴリズムを考案したことは比較的良く知られているが、
    このような文章からして、
    数学に関係なく政治に興味を持っていたことが分かる。
    ここでキャロルは、勝馬に乗りたいという人々の欲望や
    付和雷同を導くような選挙方式も、贈収賄と同じく不正であると糾弾し、
    盲目的な集団の情熱ではなく個々人の力への信頼をうたう。
    }

\section*{}

セント・ジェイムズ・ガゼット編集者様:

\bigskip

 拝啓。ユートピアとは喜ばしく、よく秩序づけられた国であり、
我等のこの小さな島では見られぬ多くの恩寵にあふれております。
その中には、疑いもなく、我々が永遠に諦めざるをえないものもあります。
(例えば、ユートピアでは誰一人として夕食会で退屈することはなく、
タクシー運転手に不当な運賃を請求されることもないのです。)
しかし、その他のものは、努力によってまだ、
我々のものにする希望を持てるかも知れません。
これらの到達可能な素晴しきものの中で、
何より貴重に思われるものは「純正なる選挙」でありましょう。
ユートピアの投票者は(あまりに陳腐な事実に言及することをお許しいただきたい、
しかし我々は議論の出発点として前提が必要なのであります)、
その時の政治的問題について独立した意見を持つに十分な教育を全員が受けています。
そして、彼等は自分自身の意見どおり、何の恐れもえこ贔屓もなく投票するのです。
これが望むべき、そして目指すべき状態であることを、
そして嫉妬深いパルカ
たち\footnote{訳注:ローマ神話での運命と出産の女神たち}
がその完全な達成を許さないとしてもまだ、
それに近づかんとすればするほど、我々はより良く、より幸せになるだろうことを、
あえて否定する人はいないでしょう。
では、これを我々の目標とするとき、
道を阻む主な障害は、不正なる選挙の主な源泉は、何でしょうか?

\medskip

私が思うに、贈収賄が最初に挙がるでしょう。
ヤコブがスープを売りエサウが長子権を売った運命の日より
\footnote{訳注:旧約聖書「創世記」第25章。
ヤコブは空腹の兄エサウに、スープと交換に長子権を譲るよう持ちかける。}、
この狡猾な毒は社会の血脈の中でうずき続けています。
しかし、不正な影響はどれも、
投票者自身の自由な判断という、
国にとって最善なもの以外の理由による投票をうながすものとして、
程度の差こそあれ、贈収賄と同じであります。
私が「不正な」\footnote{訳注:原文は ``corrupt".}と言うときには、
単に自分自身の意見を持つ能力に欠ける無教育な選挙人が、
必然的に不正な投票をする、と主張しているのではありません。
このような人もともかく投票すべきであることは、
良き国のためにならないかもしれないし、
私自身はためにならないと思うのですが、
それでも、投票する本人の動機は純粋なものです。
例えば、候補者の一人が自分の個人的な知り合いであり、
立派な人柄だと知っていて投票する場合、
政治家は「暮し正しければ、誤ちもおかさず」という言葉は、
(論理的には)根拠が薄いかも知れませんが、(倫理的には)不正ではありません。
さらに、投票者は自分より賢い友人の助言の影響を受けて投票するかも知れません。
これらは高尚な動機ではありませんし、明白に政治の範囲外のものですが、
今ここで考えている悪、
つまり不正な投票という大いなる悪をなしてはいません。
しかし、金銭\footnote{訳注:原文では ``£ s.d".
ヨーロッパ、特にイギリスで十進法以前の貨幣を表す語。
ラテン貨幣単位、librae, solidi, denarii の省略形が語源。}
の言葉では表現されない贈収賄というものが存在します。
つまり、黄金を提供しても気持を動かせない多くの人々も、
世論の流れについていくために政治的信念を曲げるでしょうし、
誘惑に打ち勝つ少数者に忠誠を尽すよりも、
勝利の叫びを噴出させようと自分の声を添えることを選ぶでしょう。

\medskip

どちらの形の贈収賄も今日の開かれた選挙にはびこっています。
無記名投票用紙による投票の導入は、期待通り、
おおいに贈収賄を減少させてきました。
不正直者は自分の一票に高い値段をつける機会が少なくなっており、
実際、金と引き換えに投票してきたことも証明できません。
また、どんなに望もうとも、勝利する側に確実につくこともできません。
というのも、多くの選挙においては、
どちらの側が勝つのか、全てが終わるまで誰も知らないからです。
しかし、個別の投票者に対してはその影響を減少させられたとしても、
この邪悪な影響、つまり、勝利する側につきたいという情熱は、
有権者層全体について言えば、衰えぬ活力で繁栄し続けています。
そして、この形の贈賄こそ、私が注意を引きたいと思っているものなのです。

\medskip

 1874 年と 1880 年の総選挙を思慮深く観察した者ならば、
ひとたび流れが方向を決めてしまうと、
それはますます嵩を集めながら自らを運んで行き、
あふれる川に投げこまれた一本の藁のように、
後の選挙の結果を決めてしまう姿に驚くことはありますまい。
選挙投票の初日や翌日の間は、各々のささやかな支持者たちは、
自分自身を全体結果の中の独立した因子であると感じていました。
つまり、流れを大きくすることにも、堰き止めることにも、
何か実質的なことができるだろう、と。
しかし、総選挙が終わるずっと前に、
戦いの勝敗は事実上、決まってしまっていたのです。
打ち負かされた政府はその陣営をたたみつつあり、
対立党は政権を乗っ取れる圧倒的多数に有頂天になっていました。
そして、最後の数日間に支持者を選んだ不幸な投票者は、
自分たちの役割はもっとつつましいものだと知ることになりました。
問題はもはや、「どちらの政策が国家にとって良いものか?」ではなく、
「自分にとってどちらの立場が良いだろう?
勝利の流れを大きくするか、それとも、
望みなく打ちのめされた少数者に一票を加えてひっそりしているのか?」
だったのです。

\medskip

 しかし、これが全てではありません。
この悪は多勢の情熱の高波に押し流された一つの支持層より、
はるかに広がっていきます。
いいえ、一つの総選挙よりも広がり、
我々の国家の歴史の一つの特徴すら構成するのです。
これは、イギリスの未来の暗く不吉な前兆であります。
総選挙が現状のように行われる限り、
1874 年と 1880 年のような政治権力の振り子運動に陥りがちになり、
さらに暴力性は増加するでありましょう。
つまり、一つの政党の全くなすがままの国会、
次には他方の政党の全くなすがままの国会、という具合です。
一時の政府は、その前の政府がしたことを急ぎ喜んで反故にし、
あまりに巨大な権力をふるうので、
全ての課題が破滅することになるでしょうし、
討論は笑劇となることでしょう。
一言で言えば、我々は目先の必要だけで満足して暮らす国家となり、
何の定まった原則もない、
「右向け、右!」だけが行進命令になる軍隊になってしまうでしょう。

\medskip

この悪の存在を認識している人々、
そして全ての有権者が総選挙の最初の日に投票する人々と同じく自由に候補者を選択して投票できることが望ましいということを認める人々に対して、
単純な政治的治療法を提案させてください。
それは、総選挙が終了するまで各選挙の結果を秘密のままに保っておくべきだ、
ということなのです。
投票箱を政府当局者によって封印して、
不法な干渉が不可能なような保管場所においておくようにすることには、
何の実質的な困難もないことは確かです。
そして、最後の選挙投票が終了したときに、
投票箱を開けて、投票用紙を勘定し、結果を発表してはどうでしょうか。
次のことを指摘しておくのは価値があるでしょう。
私が考えている特定の悪、
つまり、全てが終了する前に結果を発表することによって成される害に関しては、
秘密投票\footnote{訳注: ``The Ballot Act".
イギリスで 1872 年にウィリアム・グラッドストンによって
議会に提出、可決された選挙方式。}のような単純選挙と、
いまだに続いている総選挙との間に、正確な対応関係があるということです。
つまり、一方の「投票者」には、他方の「支持層」が対応しているのです。
私の提案は、実際、秘密投票から来る長所を、
これは既に単一の選挙においてはもたらされていますが、
その集合体である総選挙にも拡大しよう、ということです。

\medskip

結論として、
既に受け入れられている原則のこの新たな応用に反対するかも知れない方に一言、
言わせてください。
次のような声も聞こえてきます。
「これはまさに、ユートピア的な方法じゃないか!
こんな素晴しい意見は現実の世の中の厳しい流儀には耐えられはしませんよ!」。
しかし、そう言う人々は心の奥底では密やかにこう言っているのです。
「私は総選挙方式の楽しい一時の方が好きだね。
というのも、私のお気に入りの政党が、
高波に押し流されてしまうはずの他党よりも勝つ可能性が高いから」。
私はこのような人たちにこう言いましょう。
「私がここに書いてきたことは、
あなた方のような人に向けたものではないのです。
あなたと私とは議論するための共通の前提を持っていません。
私達の間ではどんな議論も不可能です。
私は、我々が生きているこの世界が個々人の力で成り立っているものであって、
盲目的な力によるものではないということ、そして、
我々一人一人の存在がその世界から、
力と呼ばれるもの全てを受け取ってきたこと、
我々一人一人の存在がその世界に対し、
委ねられた力の使い方に最終的な責任を負うものであるということ、
これらの古風な考え方に執着し続ける、
ささやかで無教養な、社会の一隅の人々に向けて書いたのです」、と。

\bigskip

貴殿の忠実なるしもべ、
ルイス・キャロル
4 月 30 日

\end{document}
