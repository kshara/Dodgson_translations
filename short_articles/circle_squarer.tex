% A Japanese Translation of "Simple Facts about Circle-Squaring" Chapter 1
% by Keisuke HARA
% Encoding: utf-8

\documentclass{article}

\title{「円積問題についての単純な事実」 第一章
    \footnote{この翻訳文書およびそのソースコードは
    Creative Commons CC BY 2.1 JP に従う。
    すなわち、著作者(原啓介)のクレジットを表示する限りにおいて、
    自由に本作品を複製、頒布、展示、実演、二次的著作物の作成が可能で、
    また、本作品を営利目的で利用することができる}
    \footnote{pdf ファイル https://dl.dropboxusercontent.com/u/3255303/circle\_squarer.pdf}
    \footnote{ソースコード https://github.com/kshara/Dodgson\_translations/}
    \\
    \smallskip
    {\large
        ``Simple Facts about Circle-Squaring" Chapter 1\footnote{
            First printed in {\it Lewis Carroll Centenary in London,
            Special Edition},
            ed. Falconer Madan, London, J. \& E. Bumpus, (1932), pp.122-125.}
     \\
    (by Charles L.Dodgson, M.A.)
    }
}
\author{訳: 原 啓介}

\begin{document}

\maketitle

\abstract{訳者による要旨:与えられた円と同じ面積を持つ正方形を作図せよ、
という「円積問題」を解決したという素人数学者たちに嫌気がさしたのだろう、
円の面積に近くないものは直ちに却下してしまえ、という主旨で、
$3.1413 < \pi < 3.1417$ を初等的に示そうとした書籍
(部分的な原稿は残っているが、出版はされず)の序文。
ちなみに、円周率が超越数であることを証明した F.Lindemann の論文が
Math. Annalen, Vol.20, Issue 2 に掲載されたのは 1882 年であり
(雑誌の出版日付は 6 月 1 日)、
ドジソンがこの文章を書いたのは同年 4 月 20 日とされているから、
ドジソンが円積問題の不可能性を事実として知っていたかは、微妙な問題である。
}

\section{はじめに}
\begin{flushright}
1882 年 4 月 20 日
\end{flushright}
 ワーテルローの戦いの細部についての論争が起こったとしましょう。
ある研究会で、ビューロー宰相のプロシア軍が戦場に姿を現したのは正確にいつだったか、
という問題が提起されたとして、
それが午後 6 時の少し前か、または少し後だったとする理論を主張する者は、
疑いもなく、皆に辛抱強く話を聴いてもらえることでしょう。
しかし、それが7月『19日』の午後 4 時だったことを証明した、
と主張する者に対して学会の聴衆は何と言うでしょうか。
彼等は声を揃えてこう叫ぶに違いありません。
「もし何よりも確かな歴史上の事実があるとしたら、
その戦いは『18日』に起こったということなのですよ。
その日の境を踏み出してしまうことは、単に馬鹿馬鹿しいだけです。
私達はその問題の正しい『データ』を認めない人の話を聴いて時間を無駄遣いできません」。

まさにこれが「円正方形化人」(``Circle-Squarers")たち、
この言葉で私は円の面積の『正確な』値を、
その半径を一辺の長さに持つ正方形で表そうと試みた全ての人を指しているのですが、
その彼等の理論に対する私の立場なのです。
数学会は問題の値がその正方形の面積の 3.14159 倍のあたりだと一致して認めております。
実際、上に書いた値に非常に近く、この値では小さ過ぎ、
一方 3.1416 では大き過ぎることが分かっています。
ですから、それがこの値より少し大きいか小さい値、
例えば、3.14161 とか 3.14158 であるとかいう理論を提案する人には、
おそらく耳を傾ける人がいてくれるかも知れませんが、
それが 4 と 1/2 であることを証明したと主張する理論家に対しては何と言うでしょうか?
「申しわけありませんが、」 — と、我々は叫ぶことでしょう —
「もし何よりも確かな幾何学上の事実があるとしたら、
それは円の面積はその円に内接する正方形の面積よりも大きく、
外接する正方形の面積よりも小さい、ということなのですよ。
そして、これら二つの正方形の面積はそれぞれ円の半径の 4 倍と 2 倍なのです。
これらの境を踏み出してしまうことは、単に馬鹿馬鹿しいだけです。
私達はその問題の正しい『データ』を認めない人の話を聴いて時間を無駄遣いできません」。

円正方形化人たちの理論がみな、
円の面積がその半径を一辺の長さに持つ正方形の面積の 4 倍より大きい、
あるいは 2 倍より小さい、と主張するものならば、上に述べたことで十分でしょうし、
この小著を書く必要はなかったでしょう。
しかし、彼等が提唱する数はこれほど離れているとは限りません。
ですから、そういう提唱者たちにこの論法で回答するには、
その限界の数字の組は 4 と 2 よりも、ずっと近いものでなければなりません。

 そしてこれこそ、私の思いついたことで、これは可能なのですが、
数学における極めて単純な事実、
それを否定するのは 2 と 2 を足すと 4\footnote{F.Madan によれば、
ドジソンのオリジナル原稿には、4 ではなくて 5 と書かれているらしい。
ドジソンのユーモアかも知れないが、誤解を招かぬよう訂正しておく。}
であることを否定するのとまさに同じくらい、
単純な事実しか用いない方法なのです。
円の面積を測ること自体は複雑な問題であり、
その過程も、それによって 3.14159 という値が計算されるのですが、これも長く難解なものです。
そして、どんな円正方形化人も、もしその結果を却下するために呼び出したならば、
対抗する理論に耳を傾けてくれる以前に、非常にもっともな次のような言葉を返すことでしょう。
「時間の無駄遣いをしないという点からすれば、
私がそんな複雑な計算をマスターするのに何ヶ月も、それどころか何年も費すよりも、
そう言う『あなた』こそ、私の理論を検証して、もしあなたに出来るならですが、
それを反証することに数分間くらい使うべきでしょうに」、と。

 「では、これではいけませんか」、と次のように言われるかも知れませんね。
「あなたが出会う円正方形化人それぞれの定理を単に反証することで満足しては?
彼等の証明はふつう短くて、初歩の幾何学の範囲を越えることは滅多にないのですし、
我々が良く知っているように、常に間違っていて、
簡単に分かる論理的誤ちを必ず含んでいるのですから」。

 まさにその通りでしょう。しかし、まず第一に、
円正方形化人のお気に入りの定理の反証こそ、
彼が我慢して聴くことすら全くありえない、まさにそのことなのですよ。
長く考え続けたその骨折り仕事の結果、
本人はその定理が自分の存在と同じくらい確かなものだと、
信じるようになっているのです。
そして第二に、新たな円正方形化人が現れるたびに、
新たな論法が必要になるかも知れません。
その代わりに私がこの本で説明しようとしていることは、
全ての円正方形化人に等しく適用できる回答なのです。

私がとる道筋は短く言うと、こうです。
まず、あとで使う初歩的な事実のリストを与えます。
そして次に、非常に単純な方法を用いて
(ここでは『円』を計測しようとしたりは全くせず、
円の内側や外側に描かれた直線でできたある種の図形を測るだけです)、
円の面積の『正確な』値が何であろうと、それはとにかく、
円の半径を一辺とする正方形の面積の 3.1417 倍より小さく、
3.1413 倍より大きい、ということを証明します。
ゆえに、この二つの値の間から外れた面積を主張する円正方形化人については、
この小著が十分な回答になっているだろうと期待します。
彼はここに挙げる証明が、理解するのに長過ぎるとか、難し過ぎるとか、
抗弁することはできますまい。
そして、世界中のほとんどの人の意見と矛盾する定理の話を聞いてもらおうとする前に、
公平にも、まずは上に挙げた二つの値の真実を反証せよと言われるでしょう。
もし彼が、私がこの本に挙げる初等的な真実のどの一つでも除外するなら、
もちろん法廷からただちに退場していただきます。
それらの真実は 2 足す 2 が 4 であるのと同じくらい確実な足がかりに立っているのですから、
それ以上の議論は時間の無駄遣いというものです。
しかしながら、彼がこれらの初等的な真実を認めるのならば、
逆らうことのできない論理によって、
上に挙げた二つの数字の範囲を受け入れることは避けられません。
これらを導くのに用いた方法は、彼自身で容易に推し進めることができ、
新たな数字の対を見つけられるものです。
つまり、もし彼が 3.1417 と 3.1413 より少し内側に入る値を主張していても、
その値が正しいという可能性を否定する新たな範囲を、自分の手で見つけられるでしょう。

 $\pi$
(通常、円の面積の、その半径を一辺に持つ正方形の面積に対する比に与えられる名前)
の正確な値は、あらゆる時代を通じて人を迷わせてきた「鬼火」であり、
数百人もの、もし数千人でなければですが、
長年虚しく探し続けられてきた発見をして彼等の名を不滅のものとするという望みの下、
不運な数学者たちに貴重な年月を無駄遣いさせてきました。
私はこの小著が、まやかしの光に惑わされている人々の手に渡り、
そうでなければ無駄に費されるだろう時間と労力から救い出せることの期待を、
胸に抱いているのであります。


\end{document}
