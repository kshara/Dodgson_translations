% A Japanese Translation of "The Matrix Cipher" (C.L.Dodgson)
% by Keisuke HARA
% Encoding: utf-8

\documentclass{article}

\title{行列暗号
    \footnote{この翻訳文書およびそのソースコードは
    Creative Commons CC BY 2.1 JP に従う。
    すなわち、著作者(原啓介)のクレジットを表示する限りにおいて、
    自由に本作品を複製、頒布、展示、実演、二次的著作物の作成が可能で、
    また、本作品を営利目的で利用することができる。}
    \footnote{pdf ファイル https://dl.dropboxusercontent.com/u/3255303/matrix\_cipher.pdf}
    \footnote{ソースコード https://github.com/kshara/Dodgson\_translations/}
    \\
    \smallskip
    {\large
        ``The Matrix Cipher"\footnote{
            {\it The Lewis Carroll Circular}, No.2,
            Trevor Winkfield, ed. (Leeds: privately printed, Nov. 1974),
            pp.67--68.}
     \\
    (by Charles L.Dodgson, M.A.)
    }
}
\author{訳: 原 啓介}

\begin{document}

\maketitle

\abstract{訳者より:
    ルイス・キャロルことドジソンが日記に書き残した
    二つの暗号方式の後者で(1858 年 2 月 26 日付)、
    ドジソンは「前のものよりずっと良い」としている。
    文字の対応表とキーワードを組み合わせた多表式暗号の変種だが、
    もちろん現代の実用に耐えるものではない。
    しかし、アルゴリズム的にも数学史的にも興味深く、例えば、
    Lipson-Abeles (1990)は「暗号の見地からのみならず、
    19 世紀イギリスの数学の水準からしても、歴史的価値がある」として、
    現代的な立場から考察を与えている\footnote{S.H.Lipson and F.Abeles,
        {\it ``The Matrix Cipher of C.L.Dodgson"},
    Cryptologia, vol.14, no.1, (1990), pp.28--36.}。
    }


\section*{}

前のものよりもずっと良い、別の暗号を発明した。
これは以下の利点を持つ。
(1) このシステムは簡単に暗記できる。
(2) 秘密にしておくべきものはキーワードだけである。
(3) このシステムを知っているものでさえ、
キーワードを知らないことにはこの暗号を解けそうにない。
(4) 与えられた暗号文からキーワードを見つけることは不可能である。

アルファベットを以下のように並べて書く\footnote{通常のアルファベットでは
5 行 5 列の 25 文字を越えてしまうので、
``J" と ``U" を用いないラテン語を用いたのだろう。}:

\[
\begin{array}{ccccc}
    A & F & L & Q & W\\
    B & G & M & R & X\\
    C & H & N & S & Y\\
    D & I & O & T & Z\\
    E & K & P & V & \ast
\end{array}
\]

キーワードを ``ground" とし、送るメッセージの最初の単語を ``send" としよう。
G から S へと辿って、
「2 番目の列、1 番目の行」となるので、
21 と書く\footnote{上のアルファベット表で、
``G" から 2 列右、1 行下の位置に ``S" がある、の意味。}。
この数字から再現するには、G から始めて「右に 2 列、下に 1 行」進んで、
S が得られる。
R から E へと測って 23 を、
O から N へと測って 04 を、V から D で 24 を得る。
すなわち、``send" の暗号として 21. 23. 04. 24 と書く。

ある文字から別の文字へ数えるときは、
鍵になる文字が入っている列を第 0 列と考え、
最後の列までくると 1 列目に戻る。
行も同様である\footnote{基本的な暗号システムの説明はここまでで終わりで、
以降は良く言えば実践的な、悪く言えばアドホックな、複雑化の方法あれこれ。
この色々な追加的なトリックは、
現代の数学的暗号理論の立場からすればおそらく興味を引かないが、
私見ではキャロル的な部分である。
}。

偶然に文字を書き落としたり、繰り返してしまったりすると、
メッセージが混乱することになるので、これを改善するため、
時に数字を括弧の中に書き、鍵のどの文字が届いたのかを示す。

もし鍵として二つ以上の単語がある場合には、
括弧の中に二つの数字を書き、
何番目の単語であるか、その単語の何番目の文字かを示す。
もし鍵が一つの単語だけである場合には、それでも、
一つ目の数字をランダムに書く。
例えば、与えられた鍵の文字 ``n" に達したとき、
(2.5.) と挿入してもよい。
これによって、鍵に二つの単語があるように見える。
または、(7.11.)と挿入するかも知れない。
``ground" の 11 番目の文字を知るには、
もちろん語を二度繰り返して文字を数えるのだが、
より簡単な方法としては 6 で割った余りをとればよい。
もし D よりあと他の文字で続けたいときには、
二番目の括弧を使おう。
例えば、(2.5.) (1.2.) は「R で続ける」の意味である。

また各段落を不必要な文字いくつかで始めたり終えてもよく、
これらはメッセージを受け取る側が単に却下する。
何文字が各文末にあるかを示すために、
その始まりの括弧にアルファベットを足して書く。
例えば、(1.2.) Q と書くと、我々は Q を 2 番目の鍵文字、
つまり R を使って解釈し、「0列、1行」と読める。そして、
これは「段落の始まりの文字は却下しないが\footnote{つまり、0文字を却下。}、
最後の 1 文字は却下する」と解釈する。

もし、これとは別に、我々が文字を追加したり書き落としてしまったりして、
単語を書き間違えたならば、
暗号と対応する英文から、鍵を発見することは不可能だろう。

例として、``ground" を鍵として、メッセージを ``send him here" としよう。
これは例えば、次のように暗号化される。
(2.3) (V) 10. 14. 20. 00. 00. 01. 33. 40. 42. 40. 01. 20. 23. 02.
これを復号するには、
三番目の鍵文字 ``O---" から始め\footnote{(2.3) の意味は、
「キーワードの 2 語目の 3 番目の文字を鍵文字とせよ」。
今の場合はキーワードが ``ground" の一語だけなので、
この語の 3 番目の文字 ``O" から鍵文字が始まる。}、
``O" から ``V" へ辿って ``1.1" を得る。
つまり、我々は 1 番目の文字と、最後の文字を却下して、
その他を変換する\footnote{(V) の意味は「この文字を解釈した数字に対応する
文字群を却下せよ」。今、鍵文字は ``O" だから、``V" は 1 行目の 1 列目で
``1.1" となり、文頭の 1 文字と文末の 1 文字を却下する。}。
よって、O.14 から始まって\footnote{最初の文字 ``10" は却下し、
次の ``14" から開始。最後の ``02" も却下。}、
V.20 とその後を続ける\footnote{``ground" の ``O" の次は ``U" だが、
アルファベット表に ``U" はないので ``V" が使われる。
これはラテン語で ``U" を ``V" で、``J" を ``I"
で表記することによると思われる}。
ゆえに、我々は ``send h[k]im here" を得る。
通常、我々は ``k" を書かないで、出てきたときにはそれを除く。

これをさらに改善するには、1.4 の代わりに
D (第一行の第 4 文字である)と書く、などである。

\end{document}
