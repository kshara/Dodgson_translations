% A Japanese Translation of "Something or Nothing?" and etc. (C.L.Dodgson)
% by Keisuke HARA
% Encoding: utf-8

\newtheorem{prob}{問題}
\renewcommand{\theprob}{}

\documentclass{article}

\title{「何かある?何もない?」と「問題9588番について」
    \footnote{この翻訳文書およびそのソースコードは
    Creative Commons CC BY 2.1 JP に従う。
    すなわち、著作者(原啓介)のクレジットを表示する限りにおいて、
    自由に本作品を複製、頒布、展示、実演、二次的著作物の作成が可能で、
    また、本作品を営利目的で利用することができる}
    \footnote{pdf ファイル https://dl.dropboxusercontent.com/u/3255303/something\_or\_nothing.pdf}
    \footnote{ソースコード https://github.com/kshara/Dodgson\_translations/}
    \\
    \smallskip
    {\large
        ``Something or Nothing?"\footnote{
            {\it Mathematical Questions and Solutions from "Educational Times"},
            Vol. XLIX,
    W.J.C.Miller, ed. (London, Francis Hodgson, 1888), pp.101--102.}
    (by Charles L.Dodgson, M.A.)
    and ``Question 9588"\footnote{
        {\it Mathematical Questions and Solutions from "Educational Times"},
    Vol. L, pp.34--35.}
    }
}
\author{訳: 原 啓介}

\begin{document}

\maketitle

\begin{abstract}
訳者による要旨:
「線分上にでたらめに一点を選ぶとき、それが事前に指定した点と一致する確率は?」
という問題に対する論争。答は $0$ であると主張する派と、
「ある種の無限小」であると主張する派に分かれ、
ドジソン自身は後者だった。
ドジソンは前者を否定する目的で、
その点が有理数である確率と無理数である確率の両方を問うた。
もし最初の答が $0$ なら、このどちらの答も $0$ であるはずで、
その合計が $1$ でなくてはならないことに矛盾する、と言うのである。
公理的確率論と測度論を知っている現代の我々から見れば、
ドジソンのロジックは誤りで、滑稽な印象さえ与えるが、着眼点は鋭い。
この論争が掲載された ``Educational Times"
はシリアスな学術雑誌ではないとは言え、歴史的にも興味深いと思う。
\end{abstract}

\section*{何かある?何もない?}

  1885 年と 1886 年に、問題 7695 番
(Vol. XLIII., pp.86 と XLIV., pp.24 を参照のこと)
の解答に関して、偶然の理論における困難について議論があり、
典型的な問題は次のようなものでした。
「与えられた直線上にでたらめに一点を選ぶとき、
それが事前に指定した点と一致する確率はいくらか?」。
一方はその確率は【全くのゼロ】であると主張し、
他方は、私自身もそうでしたが、
答はある種の【無限小】であって全くのゼロ【ではない】、
と主張していたのです。
当時、両側とも十分に主張を展開しましたが、
私はその議論をここに再開したいと思っています。
と言うのも、
「全くのゼロ」理論の支持者たちに、
その問題点の新たな視点を提供できると思うからです。

 私はどちらの主張をする人々も以下の公理を認めるものと仮定します。
(1)【全くのゼロ】を集めても、それが無限に多くであろうとも、
【大きさ】を持つことはできない。
(2)(上の例のように)点をいくら集めても、それが無限に多くであろうとも、
直線上にどんな長さも、それが無限に小さかろうが、持つことはできない。
ゆえに、(3)もし直線上のランダムな一点が【選んでおいた一点】である確率は
【全くのゼロ】ならば、【選んでおいた点の集まり】に含まれる確率も、
その点が無限に多くであろうとも、またそうである。

 私はここで二つの質問を投げかけたく思います。
I.「与えられた線分にランダムに一点を選んだとき、
その点が線分を【同じ長さで割り切れる】
ような二つの長さに分ける確率はいくらか?」\footnote{訳注:
「与えられた線分上の一点をランダムに選ぶとき、
この点が線分を有理数の比に分ける確率、
無理数の比に分ける確率を求めよ」の意。
現代的な確率概念の見地からすれば、
「ランダムに選ぶ」の意味を確定しなければ問題は意味を持たないが、
通常通りに一様な確率を仮定すれば、
前者の確率は $0$、後者の確率が $1$ である。
ドジソンの議論は混乱しているが、
$[0, 1]$ 区間上の有理数の集合と無理数の集合の「大きさ」(測度)
はそれぞれいくらか、という典型的なルベーグ積分の問題を、
確率の言葉で提出していることは注目に値する。}。
我々がここで【選んでおいた点の集まり】を扱っていることは明らかに思えます。
なぜなら、線分上のどんな長さを持つ量も、
ここに点が落ちれば二つの部分は同じ長さで割り切れる、
と言うことは不可能だからです。
よって、私の論敵たちは「その確率は全くゼロである」と答えるはずだと思います。

II.「では、選んだ点が線分を【同じ長さで割り切れない】
ような長さの二つの部分に分ける確率はいくらか?」。
これにもまた、彼等は「その確率は全くゼロである」と答えなければなりません。

 しかし、これらの二つの事象のどちらか一方が必ず起こるのです!
ゆえに、この二つの確率の和は数学的に1で表されなければなりません。
すなわち、どちらか一方は(どちらとは言えませんが)、
ただ「何か」があるだけではなくて、【無限小】だけでもなく、
ある不可解に高階的なものでなくてはならず、
もしそれを越えないにせよ【二分の一】という【有限の】値に、
実際とどかなくてはならないのです!

\section*{問題 9588 番について}

\begin{prob}[{\bf 8861}番 J.Brill, M.A. 出題]
与えられた長さの棒をランダムに二つに折ったとき、
それぞれの長さが同じ数で割り切れる確率を求めよ。
\end{prob}

\begin{prob}[{\bf 9588}番 Charles L. Dodgson, M.A. 出題]
与えられた線分上の一点をランダムに選んだとき、この点が線分を、
(1)同じ長さで割り切れるように分ける確率、
(2)同じ長さでは割り切れないように分ける確率をそれぞれ求めよ。\footnote{訳注:
8861番に対して、
有理数比に分ける確率と無理数比に分ける確率の二つをペアで問うたところが、
ドジソンの工夫である。この二つの事象は互いに補事象なので、
どちらか一方が起こる、つまり確率の和は 1 でなければならない。
この加法性の要請が、この問題に関する曖昧な議論の論理的矛盾を炙り出す。}
\end{prob}


\subsection*{T.C.Simmons師, M.A. による答}

 興味深い 9588 番の問題を扱うための、一つの方法は以下のようなものです。
ランダムな点 P を線分 $\lambda$ 上に取るとき、線分の両端の【上に】乗るよりも、
その両端の【間に】に落ちる方が無限に起こりやすいということは、
誰もが認めることでしょう。
今、$\lambda$ をまず半分に、次に三等分、四等分、五等分、六等分、と順に分割していきます。
この操作のどの時点においても、$n$ 個の分割の印(両端の一方の点を含めて)
が得られていきます。
点Pが与えられた任意の二つの連続した印の【間に】に落ちる確率は、
上と同様に、その二つの印の【上に】に落ちる確率よりも、
無限に大きくなるでしょう。
よって、$n$ 個の印のどれかの連続する二つの【間に】落ちる確率は、
その $n$ 個の印のどれか一つの【上に】落ちる確率より無限に大きいでしょう。
一方の確率ともう一方の確率の比は、両方の項が $n$ 倍されるので、
$n$ に独立となり、
このことはもちろん $n$ が無限大になったときも成立するでしょう。
しかし、$n$ を無限大にすると明らかに、ありうる全ての、
同じ長さで割り切れるように線分を分ける $\lambda$ の分点が、取り尽されます。
ゆえに、同じ長さで割り切れるように $\lambda$ を分ける点と P が一致しない確率は、
一致する確率よりも無限に大きくなるでしょう。
これらの理由によって、私はこの問題の答が、
(1)はゼロ、(2)は 1 になるはずだ、という【意見】で挑みたく思います。
  同じ理由によって、問題 8861 番で求められている確率もゼロになるでしょう。

\subsection*{Tanner教授, M.A. による答}

 ドジソン氏は、彼の二つの選んだ点の集まりがいかにして、
彼の公理の(2)と矛盾なく線分全体に一致するかを説明するとき
(Vol.XLIX, 2, pp.101 の "Something or Nothing" と題されたノートを参照)、
彼は、公理の(1)にも関わらず、
どのように全くのゼロの集まりが 1 になりうるかを説明することで、
「反対の主張」を助けてしまっているのです。
 記号 $\delta$ で、与えられた長さ 1 の線分上にランダムに選んだ点が、
指定した点に一致する確率を表すことにします。
点 P たちの間に幅を与え、そのとなりとは互いに $\delta / n$
だけ離れているとします。
もし $\delta$ が全くのゼロでなければ、各点 P は互いに離れています。
長さ $1/n$ の区間には、これらの点の内 $1/\delta$ 個が含まれるでしょうから、
ランダムに選んだ点が各点 P の一つ以上と一致する確率は 1 であり、
ランダムに選んだ点がこの区間の外側に落ちる確率は、
その区間がいかに短くとも、ゼロです。この矛盾を避けるには、
$\delta = 0$ でなければなりません。

\subsection*{ドジソン氏のコメント}

上の解答で用いられた論証と議論に対して、
ドジソン氏は以下のコメントを送ってくれた。

解答1について:

T.C.シモンズ師への返答として、もし、2, 3, などで線分を分割するのではなく、
$\sqrt{2}$, $\sqrt{3}$, などで分割していったなら、
そしてもし、「同じ長さで割り切れる」と書くところで、
「同じ長さで割り切れない」と書けば、彼の議論は前とまったく同様に確からしく、
問題の二つの確率はゼロと 1 ではなくて、1 とゼロであることが証明されるでしょう。
二つの矛盾する主張のどちらも等しく証明する議論には、非常に注意深い扱いが必要です。

解答2について:

タナー教授への返答として、
私が絶対に起こらないと言ったことがどのように起こりうるかを私が説明することは、
慎んで辞退させていただきます!
いかなる「点の集まり」も、私が信じるには、「線分全体をなす」
ことも、そのいかなる長さも持つこともできないのです。
よって、ご希望の説明のためにはまず、
その「反対」の説明をして下さるようお願いしたいと思います。
つまり、いかにして「まったくのゼロの集まりが 1 になりうるか」
を説明していただきたい。
もう取りかかっているのですから、
残りは大した仕事ではないでしょう\footnote{訳注:
もちろん皮肉であり、ドジソンはここに大きな困難があることは理解していただろう。
つまり、$[0, 1]$ 区間上の無理数の集合のようなバラバラの点の集まりに、
どうすれば矛盾なく自然な「長さ」を与えられるのか。
現代の我々は、
この「仕事」がルベーグによる測度概念の確立(1902年)
に他ならないことを知っている。}。
解答の後半で彼は、私が誤解していなければですが、
ランダムに選んだ一点が先に決めておいた一点と一致する確率が $\delta$ ならば、
どれか一つと一致する確率は $1/\delta$ であり、このような点たちは長さ 1 をなす、
と主張しています。
では、彼はこう言うのでしょうか。10 個の鞄があって、それぞれに
1 個の白い石と 9 個の黒い石が入っているとき、
各鞄から白石をひく確率は $1/10$ であるから、
10 個の鞄のどれからから白石をひく確率は 1 である。
彼はこれを定理の正しい例証として受け入れるのでしょうか?

\end{document}
