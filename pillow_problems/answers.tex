\chapter{答}

\subsection*{5.}
% 原文では問題番号のあとに問題と解答のページ数がついているが、最終組版に依存するので書いていない。

3 分の 2。

\subsection*{6.}

各辺の長さを $2a, 2b, 2c$, 直線の長さを $\alpha, \beta, \gamma$
と書くと、
\[
a^2 = \frac{-\alpha^2 + 2 \beta^2 + 2 \gamma^2}{9},
\]
\[
\cos A =
\frac{5 \alpha^2 - \beta^2 -\gamma^2}{2 \sqrt{2 \alpha^2 - \beta^2 + 2 \gamma^2} \sqrt{2 \alpha^2 + 2 \beta^2 - \gamma^2}}.
\]
[訳者より:キャロルは本文で $\cos$ と $\sin$ を意味する、
自分で発明した特殊な記号を使っている。
その記号を表示するのが面倒なので、
この翻訳原稿では$\cos, \sin$ と書いておく。
組版時にはキャロルの記号に差し替えるように。]

\subsection*{7.}

[訳者より:ここに図を入れる]

$AB, AC$ を与えられた両側の辺とし、
$B, D$ を直角 $\angle s$とする。
また $AB = b, AD = d$ とする。
\begin{enumerate}
\item[(1)]
\[
BC = \frac{d - b \cos A}{\sin A};
CD = \frac{b - d \cos A}{\sin A};
\]
\item[(2)]
\[
\mbox{面積} = \frac{2bd - (b^2 + d^2) \cos A}{2 \sin A}.
\]
\end{enumerate}

\subsection*{8.}

7 人。2 シリング。

\subsection*{10.}

フローリン硬貨 2 つと 6 ペンス硬貨 1 つ、
または、半クラウン硬貨 1 つとシリング硬貨 2 つ。


\subsection*{11.}

求める比は、
\[
\frac{\sin A \sin B \sin C}{\sin \theta (1 + \cos A \cos B \cos C)
 + \cos \theta \sin A \sin B \sin C}
\]
に等しい。
$\theta = 90^\circ$ のとき、この値
\[
= \frac{\sin A \sin B \sin C}{1 + \cos A \cos B \cos C}.
\]

\subsection*{12.}

$s =$ 周長の半分、$m =$ 面積、$v =$ 体積とするとき、
\[
a^2 + b^2 + c^2 = 2 \left( s^2 - \frac{v}{s} - \frac{m^2}{s^2} \right).
\]

\subsection*{13.}

中心と交点を頂点に持つ四角形の面積を $2M$ とし、
その辺を $a, b$ とし、中心を結ぶ対角線を $c$ とすると、
求める面積は
\[
 = \frac{32 M^3}{(b^2 + c^2 - a^2)(c^2 + a^2 - b^2)}.
\]

\subsection*{16.}

一番目は確率 $\frac{1}{2}$ で、二番目は $\frac{5}{12}$ である。
ゆえに、前者の方が大きい。


\subsection*{18.}

(1) 底辺 $BC$ を点 $E$ で、
$\frac{BE}{EC} = \frac{\sin 2C}{\sin 2B}$
のように分割する。

(2)
点 $B, C$ で、直角 $ABD, ACD$ を作る。
$AD$ と $BC$ が交わる点を $E$ とすると、これが求める点である。


\subsection*{19.}

17 分の 11。


\subsection*{21.}

(1)
$\frac{n (n+1) (n+4) (n+5)}{4}$;
(2)
$27,573,000.$

\subsection*{22.}

与えられた高さを $\alpha, \beta, \gamma$ と呼ぶことにして、
比
\[
\frac{2\alpha^2 \beta^2 \gamma^2 (\alpha^2 + \beta^2 + \gamma^2) - (\beta^4 \gamma^4 + \gamma^4 \alpha^4 + \alpha^4 \beta^4)}{4 \alpha^4 \beta^4 \gamma^4}
\]
を $k^2$ とおくと、

(1) $a = 1/(ka)$ など。

(2) $\sin A = k \beta \gamma$ など。

(3) 面積 = $1 / (2k)$.


\subsection*{23.}

5 分の 2.


\subsection*{24.}
\[
\frac{DO}{DA} + \frac{EO}{EB} + \frac{FO}{FC} = 1
\]
となって、どれも他の 2 つの項で表せる。


\subsection*{25.}

$\epsilon + \alpha + \lambda - 2$.


\subsection*{27.}

25 分の 17.


\subsection*{28.}

$7 - 3 \sqrt{5}.$


\subsection*{31.}

置時計が「12 時 2 分 29 と $\frac{277}{288}$ 秒」を示したとき。

\subsection*{32.}

(1) $ \frac{n(n+1)(2n + 13)}{6}$;
(2)
$358550$.


\subsection*{37.}

$\frac{4+\sqrt{3}}{12} - \frac{1 + \sqrt{3}}{2 \pi}$,
すなわち、約 0.044.


\subsection*{38.}

72 分の 49.

\subsection*{39.}

彼等は 2 日と 6 時間後、そして 4 日目の終わりに出会う。
これは出発点から 23 マイルと 34 マイルの地点。

\subsection*{41.}

(1) 12 分の 7. (2) 2 分の 1.


\subsection*{42.}
\[
\frac{abc}{2(s-a)(s-b)(s-c)}.
\]

\subsection*{45.}

$0.6321207 \cdots$.


\subsection*{47.}

値の 1 つの組は $0, 0, 0$.
 2 つめの組は $x = y = 0$ で $z$ は任意の値。
 3 つめの組は $x = z = 0$ で $y$ は任意の値。
4 つめの組は $x = k^2/(k-1), y = z = k$ で、
ここに $k$ は任意の値。
もし、$x$ が $4$ 未満の正の値なら、$y$ と $z$ は実数では存在しない。


\subsection*{49.}

$2$.


\subsection*{50.}

27 分の 17.

\subsection*{52.}

2 ポンド、18 シリング、0 ペンス。

\subsection*{53.}

二番目の辺から切り取る比は、
\[
\frac{\alpha \sin C + \gamma \sin A)(2 \gamma \cos A + \beta)}
{\alpha \cos C + \gamma \cos A + \beta}
+
\frac{\beta \cos A + \gamma \cos 2A}{\sin A}
.
\]


\subsection*{54.}

辺 $AB$ を点 $D, G$ で、
\[
AD : DG : GB = \frac{1}{a} : \frac{1}{b} : \frac{1}{c}
\]
となるように分け、他の辺も同様にする。
また、六角形の各辺の長さは
\[
= \frac{1}{\frac{1}{a} + \frac{1}{b} + \frac{1}{c}}.
\]

\subsection*{56.}

[訳者より:ここに図を入れる]

$BC, CE, ED$ を与えられた高さに等しく、
$B, C$ で直角になるようにひいて、$DB, EC$ を作る。
$DC$ を結び、それと直角に $CF$ をひく。
$EB$ を結び、それと直角に $BG$ をひく。
$B$ を中心に距離 $BF$ の円を描き、
$C$ を中心に距離 $CG$ の円を描く。
これらの交点を $A$ として、$AB, AC$ と結ぶ。
三角形 $ABC$ は求める三角形と相似であることが証明される。
構成の残りは明白である。


\subsection*{57.}

(1)幾何的に。

与えられた三角形の各辺の外側に正方形を描き、
それらの外側の辺で新しい三角形を作り、
与えられらた三角形の各辺をこの新しい三角形の各辺に相似に分割する。
これらの比を用いて求める正方形が構成できる。


(2)三角法で。

$a, b, c$ を与えられた三角形の各辺の長さとし、
$m$ をその面積とする。また $x, y, z$ を求める正方形の各辺の長さとすると、
\[
\frac{a}{x} = \frac{b}{y} = \frac{c}{z}
=  \frac{a^2 + b^2 + c^2}{2m} + 1.
\]


\subsection*{58.}
\[
\frac{3}{8 - \frac{6 \sqrt{3}}{\pi}}.
\]


\subsection*{59.}

 3 つの対の各辺の長さを $a, b, c$ とし、
各面の対応する角度を $A, B, C$ とすると、
求める体積は、
\[
\frac{abc}{6} \sqrt{1 - (\cos^2 A + \cos^2 B + \cos^2 C)
 + 2 \cos A \cos B \cos C}.
\]


\subsection*{60.}

\[
\cot BAD = \frac{(m+n) \cot A + n \cot B}{m};
\]
同様に、
\[
\cot CAD = \frac{(m+n) \cot A + n \cot C}{n}.
\]

\subsection*{63.}

各正方形の各辺 $=2$ で、体積は、
\[
\frac{8 \cdot 2^{\frac{1}{4}} (\sqrt{2} + 1)}{3}.
\]


\subsection*{66.}

\[
\frac{2^m (\alpha - \beta) + \beta}{2^m (\alpha - \beta) + 2 \beta}.
\]


\subsection*{67.}

水平面の中心を原点にとり、
その面の頂点の 1 つを通るように $X$ 軸を、
その面の対辺に平行に $Y$ 軸を、
その面に垂直に $Z$ 軸をとる。
四面体の(下向きに測った)高さをを $h$ とし、
$X$ 軸上の交点を $a$ とすると、
軌跡の方程式は、
\[
(x + \sqrt{3} y)(h - z) = ah;
\]
\[
x^2 + y^2 = a^2.
\]

\subsection*{68.}

(1) $5$ ダース。
(2) 一本 $8$ シリングと $4$ ペンス。


\subsection*{69.}

(1)
\[
k = \frac{\theta - B}{A}; \quad l = \frac{\theta - C}{B};
\quad m = \frac{\theta - A}{C}.
\]


(2)
新しい三角形を $A'B'C'$ とすると、
\[
\frac{a'}{a} = \frac{b'}{b} = \frac{c'}{c} = 2 \cos \theta.
\]


\subsection*{70.}

(1)
裏の辺を下がって、再び上がって、以下同様。
(2)
裏の辺を下がる方向に約 $2.7$.
(3)
約 $18.65^\circ$.
(4)
約 $14.53^\circ$.


\subsection*{72.}

 1 つは黒で、他方は白。

