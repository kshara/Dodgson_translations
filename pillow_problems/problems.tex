\chapter{問題}

\subsection*{1.}
% 原文では問題番号のあとに答と解答のページ数がついているが、最終組版に依存するので書いていない。

 その和=2となるような 2 つの平方の一般公式を求めよ。
\begin{flushright}
[24/3/84]
\end{flushright}

\subsection*{2.}
 与えられた三角形の底辺に平行に直線をひき、
これと底辺が両側の辺から切り取った長さの和が、
底辺の長さに等しくなるようにせよ。

\subsection*{3.}
  ある四角形の 4 辺が平行四辺形の各頂点を通り、かつ、
そのうちの 3 辺がそれら頂点で二等分されるならば、
 4 つ目の辺もそうであることを証明せよ。

\subsection*{4.}
  与えられた鋭角三角形の中に三角形を内接させ、
それらの辺が、各頂点において、
与えられた三角形の辺と等しい角度をなすようにせよ。
\begin{flushright}
[19/4/76]
\end{flushright}


\subsection*{5.}

袋に 1 つの小石が入っていて、
それが白か黒のどちらかであることが分かっている。
この袋に 1 つの白石を入れて、良く振り、
小石を 1 つ取り出したところ、白だった。
ここで次に白石を取り出す確率はいくらか。
\begin{flushright}
[8/9/87]
\end{flushright}

\subsection*{6.}

三角形の各頂点から対辺の中点にひいた線分の長さが与えられているとき、
各辺の長さと角度を求めよ。

\subsection*{7.}

 2 つの隣接する辺とその間の角度が、
ある四角形のものとして与えられていて、
この二辺の他方の 2 つの角は直角である。
このとき、(1)残りの辺の長さと、(2)面積を求めよ。
\begin{flushright}
[4 or 5/89]
\end{flushright}

\subsection*{8.}

何人かが、各々両隣を持つように円形に座っていて、
各人が何シリングかを持っていた。
一人目は二人目よりも 1 シリング多く、
二人目は三人目よりも 1 シリング多く、
と続いていた。
一人目が二人目に 1 シリングを渡し、
二人目が三人目に 2 シリングを渡し、
これを続けて、
各人が次の人に自分がもらったより 1 シリング多く隣に渡すことを、
できる限り続けた。
すると、隣りあった二人がいて、
一人は他方よりも 4 倍多く持っていた。
そこに何人がいたか?
最初に一番貧しかった者は、どれだけ持っていたか?
\begin{flushright}
[3/89]
\end{flushright}

\subsection*{9.}

一点で出会う二直線が与えられていて、
それらに挟まれた角の中にある一点が与えられている。
この与えられた点から、 2 つの直線をひいて、
それらが直交し、与えられた直線たちと、
与えられた点から二直線の交点へひいた直線が、
 2 つの等しい三角形となるようにせよ
(訳注:この「等しい」は面積が等しいの意味)。
\begin{flushright}
[11/76]
\end{flushright}


\subsection*{10.}

 3 つのポケットを持つ三角形のビリアード台があって、
それぞれの角に 1 つずつポケットがあり、
そのうちの 1 つはボールを 1 つ入れることができて、
残りはそれぞれボールが 2 つ入る。
台の上に 3 つのボールがあって、
それぞれに 1 つずつ硬貨が入っている。
台を傾けて、このボールたちを 1 つの角に転がしたが、
それがどの角かは分かっていない。
そのポケットの中身の期待値は 2 シリング 6 ペンスである。
各硬貨は何だったか?
(訳注:1 シリングは 12 ペンス。
この時代の硬貨は、1 ペニー(ペニーはペンスの単数形)、
2 ペンス、6 ペンス、1 シリング(12 ペンス)、
1 フローリン(2 シリング、つまり 24 ペンス)、半クラウン(30 ペンス)、
1 クラウン(60 ペンス) の 7 種類)。
\begin{flushright}
[8/90]
\end{flushright}


\subsection*{11.}

[訳者より: この問題中に図を入れる]

 三角形 $ABC$ がその中にもう 1 つの三角形 $A'B'C'$ を内接させていて、
$\angle BA'C' = \angle CB'A' = \angle AC'B' = \theta$
となっている。よって、内側の三角形は最初の三角形と相似である。
対応する辺の比を求めよ。
そして、$\theta = 90^\circ$ のときに解け。

 三角形は以下のように相似であることがわかる。

$\angle C'A'B' + \angle B'A'C = \mbox{$\theta$ の補角}$、

$\angle B'A'C + \angle A'CB' = \mbox{$\theta$ の補角}$。

$\therefore$ これらの対は等しい。 $\therefore \angle C'A'B' = C$.

ゆえに、$\angle A'B'C' = A$ そして $\angle B'C'A' = B$  である。

$C'A' = ka$  とする。
$\therefore A'B' = kb$ かつ $B'C' = kc$ である。
この $k$ を見つけるのが問題である。
\begin{flushright}
[31/3/82]
\end{flushright}


\subsection*{12.}

ある三角形の周囲の長さの半分と、面積と、
三角形の各辺と等しい辺を持つ直方体の体積が与えられたとき、
各辺の二乗の和を求めよ。
\begin{flushright}
[23/1/91]
\end{flushright}


\subsection*{13.}

 2 つの交差する円の各半径と、中心間の距離が与えられたとき、
交点における接線によってできる四角形の面積を求めよ。
\begin{flushright}
[3/89]
\end{flushright}


\subsection*{14.}

3 つの平方数の和の 3 倍がまた 4 つの平方数の和であることを証明せよ。
\begin{flushright}
[2/11/81]
\end{flushright}


\subsection*{15.}

もしある図形が、全ての内接する四角形の対角が補角になるようなものならば、
その図形は円である。
\begin{flushright}
[3/91]
\end{flushright}

\subsection*{16.}

  2 つの袋があって、その一方には小石が 1 つ入っていて、
それが白か黒であることが分かっている。
他方の袋には 1 つの白石と 2 つの黒石が入っている。
最初の袋に 1 つの白石を入れ、よく振り、
小石を 1 つ取り出したところ、白石だった。
ここから、以下のどちらの方が白石をひく確率が高いだろうか?
どちらの袋かを知ることなく 2 つの袋の一方から小石をひくか、
片方の袋の中身を他方に全部入れてから小石をひくか。
\begin{flushright}
[10/87]
\end{flushright}

\subsection*{17.}

与えられた三角形に対し、底辺に平行に直線をひいて、
その両端から、両側の辺に平行に底辺まで直線をひくとき、
それらをあわせると最初の直線の長さに等しくなるようにせよ。
\begin{flushright}
[3/89]
\end{flushright}

\subsection*{18.}

与えられた三角形の底辺に一点をとり、
この点から両側の辺に垂線を下ろしたとき、
その交点を結んだ直線が底辺に平行であるようにせよ。

(1) 三角法を使って。
(2) 幾何学的に。

\begin{flushright}
[11/89]
\end{flushright}

\subsection*{19.}

 3 つの袋がある。 1 つには白石 1 つと黒石 1 つが入っていて、
別の 1 つには白石 2 つと黒石 1 つが、
 3 つめには白石 3 つと黒石 1 つが入っている。
これらの袋がどの順番におかれているかは分かっていない。
そのうちの 1 つから白石をひき、
別の 1 つから黒石をひいた。
残ったもう 1 つから白石をひく確率は何か?

\subsection*{20.}

与えられた三角形の底辺に一点をとり、
そこから両側の辺まで直線を、
一方は底辺に垂直になるようにひき、
一方は左側の辺にひくとき、それらが等しくなるようにせよ。

\begin{flushright}
[5/88]
\end{flushright}

\subsection*{21.}

級数
$1 \cdot 3 \cdot 5 + 2 \cdot 4 \cdot 6 + \cdots$
を (1) 第 $n$ 項まで、(2) 第 100 項まで、足し上げよ。

\begin{flushright}
[7/4/89]
\end{flushright}

\subsection*{22.}

ある三角形の 3 つの「高さ」が与えられたとき、
(1) 三辺、(2)  3 つの角度、(3) 面積、を求めよ。

\begin{flushright}
[4/6/89]
\end{flushright}


\subsection*{23.}

 2 つの小石が入った袋があって、各々が黒石か白石であることが分かっている。
 2 つの白石と 1 つの黒石をこの袋に入れ、
 2 つの白石と 1 つの黒石をひいた。
さらに、 1 つの白石を入れて、
 1 つの白石をひいた。
今、この袋に 2 つの白石が含まれている確率はいくらか?

\begin{flushright}
[25/9/87]
\end{flushright}

\subsection*{24.}

[訳者より: ここに図を入れる]

三角形 $ABC$ の頂点から、直線 $AD, BE, CF$ をひき、
それらの交点を $O$ とする。
比 $\frac{EO}{EB}, \frac{FO}{FC}$ を使って、
比 $\frac{DO}{DA}$ を表せ。

\begin{flushright}
[5/86]
\end{flushright}

\subsection*{25.}

`$\epsilon$', `$\alpha$', `$\lambda$' を真分数とし、
ある病院の患者は、`$\epsilon$' の割合が目を失い、
`$\alpha$' の割合が腕を失い、`$\lambda$' の割合が脚を失っているとする。
 3 つとも失っている患者の可能な最小の数は何人か?

\begin{flushright}
[7/2/76]
\end{flushright}

\subsection*{26.}

与えられた三角形に対し、この内側に相似な三角形をおいて、
面積の比が 1 より小さい与えられた比に等しく、
各辺がもとの三角形の各辺に平行で、
各頂点がもとの三角形の各頂点から同距離であるようにせよ。

\begin{flushright}
[4/89]
\end{flushright}

\subsection*{27.}

 3 つの袋があって、それぞれの中に 6 つの小石が入っている。
 1 つの袋には 5 つの白石と 1 つの黒石、
もう 1 つの袋には 4 つの白石と 2 つの黒石、
三番目の袋には 3 つの白石と 3 つの黒石が入っている。
これらの袋から 2 つを選んで(それらがどれかは分からない)、
 2 つの小石をひくと、黒と白であった。
残りの袋から小石をひいたときそれが白石である確率は何か?

\begin{flushright}
[4/3/80]
\end{flushright}

\subsection*{28.}

与えられた三角形の各辺を、巡回的に、相乗平均の比で内分する。
これらの点を結んでできる三角形の面積と、
与えられた三角形の面積の比を求めよ。

\begin{flushright}
[12/78]
\end{flushright}

\subsection*{29.}

 2 つの異なる平方数の和に、 2 つの異なる平方数の和をかけたものを、
二通りの方法で、 2 つの平方数の和に書け。

\begin{flushright}
[3/12/81]
\end{flushright}

\subsection*{30.}

与えられた三角形に対し、底辺に平行に直線をひき、
その交点から底辺に向かって残りの二辺に平行に直線をひいたとき、
それらの和が最初に書いた直線の二倍になるようにせよ。

\begin{flushright}
[15/3/89]
\end{flushright}

\subsection*{31.}

7 月 1 日の私の懐中時計では午前 8 時に、
置時計では 8 時 4 分だった。
私は懐中時計を持ってグリニッジに行き、
真の時刻が 12 時 5 分であったとき、それは「正午」を指していた。
その夕べ、懐中時計が「6 時」を指したとき、
置時計では「5 時 59 分」だった。

7 月 30 日の私の懐中時計では午前 9 時に、
置時計では 8 時 57 分だった。
グリニッジで、私の懐中時計が「12 時 10 分」
だったとき、真の時刻は 12 時 5 分だった。
その夕べ、懐中時計が「7 時」だったとき、
置時計は「6 時 58 分」だった。

私はグリニッジへの各旅行の前に懐中時計のねじを巻いただけで、
どの日も一様に遅れていく。
また、置時計は常に一様に動き続けているものとする。

私はどのようにすれば、
7 月 31 日の真の正午がいつか知ることができるだろうか?

\begin{flushright}
[14/3/89]
\end{flushright}

\subsection*{32.}

級数 $1 \cdot 5 + 2 \cdot 6 + \cdots $ の、
(1) $n$ 項までの和、(2) 第 100 項までの和、を求めよ。

\begin{flushright}
[7/4/89]
\end{flushright}

\subsection*{33.}

与えられた円の中に内接し、
平行な 2 辺で 一方が他方の 2 倍であるようなものを持つ四角形で、
最大のものを求めよ。

\subsection*{34.}

与えられた一点から、2 本の直線をひいて、
一本は与えられた円の中心へ、
もう一本は 2 本の間の角度に等しい角度を含む弦を切り取るようにせよ。

\begin{flushright}
[21/12/74]
\end{flushright}

\subsection*{35.}

与えられた三角形に対して円を書き、各辺を二点で切り取って、
半径を各辺に垂直にひくとき、
それらが各辺によって与えられた 3 つの比に分かたれるようにせよ。

\begin{flushright}
[11/76]
\end{flushright}

\subsection*{36.}

与えられた三角形に対して直線をひき、
その一方の交点から他方の辺に垂直にし、
底辺とその間にはさまれた長さの和に等しくせよ。

\begin{flushright}
[3/89]
\end{flushright}

\subsection*{37.}

与えられた交差する 2 つの円は、
共通の弦が各中心と $30^\circ$ と $60^\circ$ の角度をなしている。
小さい方の円のどれだけが大きい方の円に含まれているか?

\begin{flushright}
[12/91]
\end{flushright}

\subsection*{38.}

 3 つの袋、「A」、「B」、「C」がある。
「A」には 3 つの赤い小石が、
「B」には 2 つの赤い小石と 1 つの白い小石が、
「C」には 1 つの赤石と 2 つの白石が入っている。
でたらめに 2 つの袋を選び、
それぞれから 1 つずつ小石をひいたところ、両方とも赤だった。
これらの小石は元の袋に戻し、
同じ 2 つの袋に対して同じ実験を繰り返したところ、一方は赤だった。
他方が赤である確率はいくらか?

\begin{flushright}
[3/76]
\end{flushright}

\subsection*{39.}

A と B が同じ日の午前 6 時に、同じ方向に道を歩き始め、
B は 14 マイル先から出発する。
各々は毎日、午前 6 時から午後 6 時まで歩く。
A は最初の日は、一様な速度で 10 マイルを歩き、
次の日は 9 マイル、三日目は 8 マイル、
と以下同様に歩いていく。
B は最初の日は、一様な速度で 2 マイル、
次の日は 4 マイル、三日目は 6 マイル、
と以下同様に歩いていく。
二人が出会うのは、いつ、どこでか?

\begin{flushright}
[16/3/78]
\end{flushright}

\subsection*{40.}
底角が鋭角であるような与えられた三角形の中に、
2 本の直線を、底辺に垂直に、かつ、
それらの和が頂点から底辺に下ろした垂線の長さに等しくなるように、
かつ、

(1) 頂点からひいた直線から等距離であるようにひけ。

(2) 底辺の端点から等距離であるようにひけ。

\begin{flushright}
[5/76]
\end{flushright}

\subsection*{41.}

私の友達が、それぞれ黒石か白石の 4 つの小石が入った袋を持ってきた。
彼が私に 2 つの小石をひくように言い、ひいてみると、
その結果は両方とも白石だった。
そこで、彼が言うことには、
「君が始める前に、
袋の中には少なくとも 1 つは白石が入っている、
と教えるつもりだった。
しかしながら、私が言うまでもなく、今、君はそれを知ったわけだ。
では、もう 1 つひきたまえ」。

(1) 私が今、白石をひく確率はいくらか?

(2) もし彼が何も言わなかったとしたら、その確率はいくらだったか?

\begin{flushright}
[9/87]
\end{flushright}

\subsection*{42.}

与えられた三角形のそれぞれの角を二等分し、
かつ、二等分線に対して垂直で各頂点を通る直線をひくと、
新しく三角形ができる。
与えられた三角形とこの三角形の面積の比を求めよ。

\begin{flushright}
[17/5/78]
\end{flushright}

\subsection*{43.}

与えられた三角形の底辺の両端から、両側の辺に直線をひいて、
同じ底辺を持つ二等辺三角形をつくり、
かつ、それを頂点にある四角形と等しくせよ。

\begin{flushright}
[2/82]
\end{flushright}

\subsection*{44.}

$a, b$ を互いに素な二数とするとき、
$(a^n - 1)$ が $b$ の倍数になるような $n$ が見つけられる。

\begin{flushright}
[18/3/81]
\end{flushright}

\subsection*{45.}

無限個の棒を折ったとき、少なくとも一本は真ん中で折れている確率を求めよ。

\begin{flushright}
[5/84]
\end{flushright}

\subsection*{46.}

その底辺に一点が与えられている三角形に対し、
その中に三角形を内接させて、
それらの角度が与えられた角度に等しく、
かつ、与えられた一点で指定した頂点を持つようにせよ。

\begin{flushright}
[19/11/87]
\end{flushright}

\subsection*{47.}

以下の連立不定方程式を解け。
\[
\left\{
\begin{array}{ll}
\frac{x}{y} = x - z; \quad \mbox{(1)}\\
\frac{x}{z} = x - y; \quad \mbox{(2)}
\end{array}
\right.
\]
そして、その実数の値が存在する範囲を、もしあれば、求めよ。

\begin{flushright}
[12/90]
\end{flushright}

\subsection*{48.}

与えられた三角形の各辺の外側に半円を描き、
そしてそれらの共通接線をひいて、
その長さを $\alpha, \beta, \gamma$ とするとき、
\[
\left(
\frac{\beta \gamma}{\alpha} + 
\frac{\gamma \alpha}{\beta} + \frac{\alpha \beta}{\gamma}
\right)
\]
が三角形の周長の半分に等しいことを証明せよ。

\begin{flushright}
[9/2/81]
\end{flushright}

\subsection*{49.}

 4 つの正三角形が、底面が正方形のピラミッドの側面をなしているとき、
その体積とこれらの三角形から作られる正四面体の体積との比を求めよ。

\begin{flushright}
[16/11/86]
\end{flushright}

\subsection*{50.}

 2 つの袋、H と K があって、
それぞれ 2 つの小石が入っている。
そして、その小石はそれぞれが黒石か白石であることが分かっている。
 1 つの白石を袋 H に追加して、その袋をよく振り、
 1 つの小石を取り出して(その色を見ることなく)袋 K に移した。
同じことを繰り返して、今度は袋 H の方に小石を移した。
いま袋 H から白石をひく確率はいくらか?

\subsection*{51.}

与えられた三角形の一方の側辺の上の与えられた一点から、
もう一方の側辺まで直線をひいて、
その両端点から底辺に垂線を下ろすと、
その和が最初の直線に等しいようにせよ。

\begin{flushright}
[12/81]
\end{flushright}

\subsection*{52.}

5 人の乞食が輪になって座り、それぞれその日の稼ぎを、
自分の前に山にしていた。
その 5 つの山は同じ大きさだった。

そのとき、その中で一番年上で、一番賢いものが、
空の袋を開けて、こう持ちかけた。

「私の友人たち、君達にちょっと面白いゲームを教えさせてくれたまえ。
まず、私は自分に「1 番」と名前をつけ、
私の左隣を「2 番」、順番に「5 番」まで名付けることにしよう。
そして、私はこの袋に今日の稼ぎを全部入れて、
それを左隣を 1 つ飛ばしてその次の者に、つまり「3番」に渡す。
このゲームの自分の番になったら、
その者は袋の中から硬貨を取り出して、
その両側の者に、彼等の名前と同じだけの枚数を渡す
(つまり、彼は「4 番」に 4 枚、「2 番」に 2 枚渡さなければならない)。
そして次に彼は受けとった時に袋の中に入っていた枚数の半分の数の硬貨を、
自分の山から袋の中に追加して、また左隣を 1 つ飛ばしてその次の者に、
つまりもちろん「5 番」だが、その者に渡す、
彼も同じように続けて、これを「2 番」に渡し、
次は「4 番」に渡し、そしてまた私に戻ってくる。
もし、自分の山の硬貨では間に合わなくなった者は、
他の者の山から自由にとって補ってもよい。
ただし、私の山だけは駄目だ!」。

他の乞食たちは大いに熱狂してこのゲームに参加した。
そのうち、袋は「1 番」のもとに戻ってきて、
彼はそのゲームの間に受け取った 2 枚の硬貨を袋に入れて、
注意深く袋の口を紐で結んだ。
そして、「これは非常に面白い、ちょっとしたゲームだね」
と述べると、
立ち上がって、急いでそこを去ってしまった。
その他の 4 人の乞食は哀しげな顔つきで互いに見つめあった。
彼等の誰にも 1 枚の硬貨も残っていなかったのだ!

各人が最初に持っていたのはいくらだったか?

\begin{flushright}
[16/2/89]
\end{flushright}

\subsection*{53.}

三角形のビリアード台上の点を、三辺の座標で与える。
与えられた一点から出発したボールが、三辺に衝突して、
出発点に戻ってきた。
このボールが二回目に衝突したときの位置を、
三辺の座標と三角形の角度で表せ。

\begin{flushright}
[6/4/89]
\end{flushright}

\subsection*{54.}

与えられた三角形から、各辺に平行な直線によって切り取って、
残りの六角形が正六角形になるようにせよ。
さらに、
与えられた三角形の各辺の長さを用いて、
この六角形の辺の長さを表せ。
また、与えられた三角形の辺が分かたれた比を求めよ。

\begin{flushright}
[18/4/86]
\end{flushright}

\subsection*{55.}

平面の上に 3 つの円柱型の塔が与えられている。
それらが同じ幅に見えるような平面上の点を求めよ。

\begin{flushright}
[20/12/74]
\end{flushright}

\subsection*{56.}

与えられた 3 つの高さを持つ、三角形を作図せよ。

\begin{flushright}
[27/6/84]
\end{flushright}

\subsection*{57.}

与えられた三角形の中に、 3 つの正方形を描き、
それらの底辺が三角形の各辺の上にあり、
その上辺が三角形をなすようにせよ。

(1) 幾何学的に; (2) 三角法で。

\begin{flushright}
[27/1/91]
\end{flushright}

\subsection*{58.}

無限平面上に 3 つの点をランダムに選ぶ。
これらが鈍角三角形の頂点をなす確率を求めよ。

\begin{flushright}
[20/1/84]
\end{flushright}

\subsection*{59.}

与えられた四面体が、どの辺も対辺と同じ長さで、
各面がどれも(外側から見て)同じであるとき、
その体積を各辺の長さで表せ。

\begin{flushright}
[8/90]
\end{flushright}

\subsection*{60.}

三角形 $ABC$ と、底辺 $BC$ を $m$ 対 $n$ の比に分かつ点 $D$ が与えられたとき、
角度 $BAD$ と $CAD$ を求めよ。

\begin{flushright}
[21/3/90]
\end{flushright}

\subsection*{61.}

3 つの数を、それらが等比数列でないように、かつ、その和が 3 の倍数であるように、
任意にとったとき、
これらの平方の和がまた他の 3 つの平方数の和で書け、
この二組には共通の数がないことを証明せよ。

\begin{flushright}
[1/12/81]
\end{flushright}

\subsection*{62.}

一点で交わる 2 つの直線が与えられ、
またこの間に挟まれた角の中にある一点が与えられたとき、
与えられた点を通る直線をひいて、
与えられた二直線と、最小の三角形を成すようにせよ。

\begin{flushright}
[12/76]
\end{flushright}

\subsection*{63.}

 2 つの等しい正方形が、異なる平行な平面の上に与えられ、
それらの中心が同じ垂直線上にあって、
その各辺が他方の対角線に平行になるように置かれており、
かつ、
隣りの頂点と結ぶことによって 8 つの正三角形ができるような距離にあるとき、
このように囲まれた立体の体積を求めよ。

\begin{flushright}
[3,4/9/90]
\end{flushright}

\subsection*{64.}

三角形とその中の一点が与えられていて、
その辺の 1 つからの距離が他の辺への距離よりも短いものとする。
与えられた点を中心とする円を描いて、
各辺から切り取った長さが直角三角形の各辺と同じ長さになるようにせよ。

\begin{flushright}
[18/12/74]
\end{flushright}

\subsection*{65.}

どの角度も $360^\circ$ を整数で割った数になっているような三角形は、
何通りあるか?

\begin{flushright}
[5/89]
\end{flushright}

\subsection*{66.}

袋の中に 2 つの小石が入っていて、
そのそれぞれが白石か黒石であることが分かっている。
また、小石を取り出してその色を見て、それを袋の中に戻す、
という実験をある回数行なった。
すると、毎回、それは白石であった。
結果として、次回、白石をひく確率は、
$\frac{\alpha}{\alpha + \beta}$ であることが分かった。
また、同じ実験をさらに $m$ 回繰り返したところ、
どの回も白石だった。
次にまた白石をひく確率はいくらか?

\begin{flushright}
[9/89]
\end{flushright}

\subsection*{67.}

正四面体が、その 1 つの頂点を下向きにして、
穴にぴったりと納められている。
これを垂直な軸について $120^\circ$ 回転し、
必要なだけ持ち上げて、ふたたび穴にぴったりと納める。
回転した頂点の 1 つの軌跡を求めよ。

\begin{flushright}
[27/1/72]
\end{flushright}

\subsection*{68.}

5 人の友人がワイン(有限)会社をすることに合意した。
各人は同じ本数のワインを出資し、それぞれ同じ価格で手に入れたものである。
次に、彼等は一人を会計係に選び、別の一人を販売係に決めて、
各ワインを原価の 10 パーセントを上乗せした値段で売ることにした。

最初の日、販売係が一本を飲み、何本かを売って、
その売上を会計係に渡した。

次の日、彼は飲みはしなかったが、
一本分の利益を着服し、残りの売上を会計係に渡した。

その夜、会計係がセラーを訪れ、
残っているワインを数えた。
彼は「11 ポンドになるぞ」とつぶやき、セラーを出た。

三日目、販売係が一本を飲み、
一本分の利益を着服し、残りの売上の会計係に渡した。

ここでワインが全てなくなった。
会社で会議を開いたところ、
彼等の利益(つまり、会計係が受け取った売上額からワインの原価を差し引いたもの)
一本あたり 6 ペンスに過ぎない、という残念な結果だった。
この三日間で日々の利益は等しかったが
(つまり、その日に会計係が受け取った売上額から、
その日に取り出したワインの原価を差し引いた金額は毎日、同じだった)、
もちろんこのことは販売係しか知らない。

(1)
彼等が持ち寄ったワインは何本だったか?

(2)
そのワインは一本いくらだったか?

(訳注:1 ポンドは 20 シリング、1 シリングは 12 ペンス)。

\begin{flushright}
[28/2/89]
\end{flushright}

\subsection*{69.}

与えられた三角形 $ABC$ のそれぞれの角を、
巡回する順序にある割合に直線で分け、
その 3 つの値は真分数で $k, l, m$ と表されているする。
そして、それらの直線によってできた三角形が、
与えられた三角形に相似ならば、
$B$ と $C$ からひかれた直線によってできる角が $A$ に等しく、
以下同様である。
このとき、 1 つの変数の関数として $k, l, m$ を表せ。
また、
第二の三角形の各辺と最初のものの対応する辺との比を求めよ。

\begin{flushright}
[8/89]
\end{flushright}

\subsection*{70.}

各辺が等しく、その角度も等しい四面体が、 1 つの面を前に置かれている。
その面に等しい三角形の列が、その面を含む平面に構成されていて、
その面と共通の底辺を持っている。
そして、それらを全て、続く限り、正四面体の周囲に巻きつける。
このとき、以下のものを求めよ。
(1)
それらの頂点の軌跡。
(2)
その左側の底角が $15^\circ$ であるものの頂点の状況。
(3)
(右に向かって巻きつけて)四面体の 4 つの面の全てを覆うとき、
「その」頂点に一致する頂点の左側の底角。
(4)
(同様にして) 4 つの面を全て覆って、
二度目にその全面とその左側の面を覆ったとき、
末端の頂点と一致する頂点の左側の底角。

\subsection*{71.}

与えられた三角形の中に、
向かいあう辺が平行で長さが等しいような六角形を置き、
それらの 3 つの辺が三角形の各辺上にあり、
その対角線が与えられた点で交わるようにせよ。

\begin{flushright}
[14/12/74]
\end{flushright}

\subsection*{72.}

袋に 2 つの小石が入っていて、
各々、黒石か白石であることが分かっている。
袋から取り出すことなく、
それらの色を確定せよ。

\begin{flushright}
[8/9//87]
\end{flushright}

